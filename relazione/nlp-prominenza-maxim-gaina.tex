\documentclass[twoside,twocolumn]{article}

\usepackage{amsthm}
\usepackage{amssymb}
%\newtheorem{thm}{Theorem}
\theoremstyle{definition}
%\newtheorem{defn}[thm]{Definition} % definition numbers are dependent on theorem numbers
%\newtheorem{exmp}[thm]{Example}

\usepackage{blindtext} % Package to generate dummy text throughout this template 

\usepackage[sc]{mathpazo} % Use the Palatino font
\usepackage[T1]{fontenc} % Use 8-bit encoding that has 256 glyphs
\linespread{1.05} % Line spacing - Palatino needs more space between lines
\usepackage{microtype} % Slightly tweak font spacing for aesthetics

\usepackage[italian]{babel} % Language hyphenation and typographical rules
\usepackage[utf8]{inputenc}

\usepackage[hmarginratio=1:1,top=32mm,columnsep=20pt]{geometry} % Document margins
\usepackage[hang, small,labelfont=bf,up,textfont=it,up]{caption} % Custom captions under/above floats in tables or figures
\usepackage{booktabs} % Horizontal rules in tables

\usepackage{graphicx}
\usepackage{float}

\usepackage{lettrine} % The lettrine is the first enlarged letter at the beginning of the text

\usepackage{enumitem} % Customized lists
\setlist[itemize]{noitemsep} % Make itemize lists more compact

\usepackage{abstract} % Allows abstract customization
\renewcommand{\abstractnamefont}{\normalfont\bfseries} % Set the "Abstract" text to bold
\renewcommand{\abstracttextfont}{\normalfont\small\itshape} % Set the abstract itself to small italic text

\usepackage{titlesec} % Allows customization of titles
\renewcommand\thesection{\Roman{section}} % Roman numerals for the sections
\renewcommand\thesubsection{\roman{subsection}} % roman numerals for subsections
\titleformat{\section}[block]{\large\scshape\centering}{\thesection.}{1em}{} % Change the look of the section titles
\titleformat{\subsection}[block]{\large}{\thesubsection.}{1em}{} % Change the look of the section titles

\usepackage{fancyhdr} % Headers and footers
\pagestyle{fancy} % All pages have headers and footers
\fancyhead{} % Blank out the default header
\fancyfoot{} % Blank out the default footer
\fancyhead[C]{Running title $\bullet$ May 2016 $\bullet$ Vol. XXI, No. 1} % Custom header text
\fancyfoot[RO,LE]{\thepage} % Custom footer text

\usepackage{titling} % Customizing the title section

\usepackage{hyperref} % For hyperlinks in the PDF

% Title Section
\setlength{\droptitle}{-4\baselineskip} % Move the title up

\pretitle{\begin{center}\Huge\bfseries} % Article title formatting
\posttitle{\end{center}} % Article title closing formatting
\title{Riconoscimento della Prominenza: Reti Neurali} % Article title
\author{%
\textsc{Maxim Gaina}\thanks{A thank you or further information} \\[1ex] % Your name
\normalsize Università degli Studi di Bologna \\ % Your institution
\normalsize \href{mailto:maxim.gaina@studio.unibo.it}{maxim.gaina@studio.unibo.it}
%\and % Uncomment if 2 authors are required, duplicate these 4 lines if more
%\textsc{Jane Smith}\thanks{Corresponding author} \\[1ex] % Second author's name
%\normalsize University of Utah \\ % Second author's institution
%\normalsize \href{mailto:jane@smith.com}{jane@smith.com} % Second author's email address
}
\date{\today} % Leave empty to omit a date
\renewcommand{\maketitlehookd}{%
\begin{abstract}
\noindent % Dummy abstract text - replace \blindtext with your abstract text
\end{abstract}
}

\begin{document}

\maketitle

\section{Introduction}
	\lettrine[nindent=0em,lines=3]{L} orem ipsum dolor sit amet, consectetur adipiscing elit.
	Text requiring further explanation\footnote{Example footnote}.

\section{Corpus}
	Nel file coi dati si ha che:
	\begin{itemize}
		\item i dati di ogni sillaba per ogni enunciato formano un unico blocco separato da una riga vuota;
		\item i dati di ogni sillaba contengono 15 feature in 3 blocchi (1 blocco per la sillaba precedente, la corrente e la successiva), sono 5 feature:
		\begin{enumerate}
			\item durata nucleo;
			\item spectral emphasis;
			\item loudness;
			\item tilt;
			\item durata sillaba.
		\end{enumerate}
	\end{itemize}
	Quindi sarà necessario scartare il primo e l'ultimo blocco (le LSTM analizzano la 
	sequenza da sole e quindi non hanno bisogno del prima e del dopo) e mantenere per ogni sillaba solo i 5 dati centrali. L'ultima colonna (0/1) indica la prominenza o meno etichettata manualmente.
	
	Dovrebbero esserci 120 enunciati. Per riprodurre gli esperimenti del paper \cite{bib:prominence-detection-italian} andranno divisi in 85/train, 15/validation e 20/test, campionandoli casualmente e ripetendo la procedura di campionamento/training/valutazione risultati per 20 volte e mediando le performance finali.
	
\section{Costruzione LSTM}
	È stato appreso quanto segue, cioè poco, e con molte domande.
	\begin{itemize}
		\item in keras va implementato un modello sequence to sequence;
		\item in input ci va una matrice (5, 34), il padding consiste di vettori nulli di lunghezza 5;
		\begin{itemize}
			\item[\textbf{\checkmark}] padding fatto con Keras
		\end{itemize}
		\item mettiamo che abbia capito chi fa il padding
		\begin{itemize}
			\item[\textbf{?}] la forma dei dati va bene o serve un embedded layer?
		\end{itemize}
		\item a questo punto, come definisco l'architettura della rete
		\begin{itemize}
			\item[\textbf{?}] quanti layer ci devono essere e di che tipo?
		\end{itemize}
	\end{itemize}
	Durante la costruzione, in futuro, provare a vedere come si comporta la rete in modalità \texttt{stateful = True} e al contrario.

\section{Lettura \cite{bib:fenomeni-prosodici-prominenza}}
	\subsection{Rassegna termini e metodi}
		\begin{itemize}
			\item definizione formale di prominenza;
			\item \textit{stress} e \textit{intonazione} (\textbf{pitch}, oppure \textbf{tono}) principali attori di collegamento fra prominenza e informazioni fonetico acustiche negli ennunciati;
			\item secondo alcuni studi il \textit{profilo del pitch} (variazione della frequenza fondamentale) risulta essere il più importante indicatore, poi lunghezza e infine intensità;
			\item teoria \textbf{pitch accent} [1958], proposta per stabilire equivalenza netta tra prominenza e fenomeni collegati all'intonazione e quindi alla configurazione assunte dalla frequenza fondamentale; però questo teoria è una presa di posizione troppo netta;
			\item teoria \textbf{metrica} [1975], creazione di albero metrico e griglia metrica, infine creazione della strutture \textit{tune};
			\item \textbf{isocronia} [1979], la prominenza è un fenomeno ritmico che avviene a intervalli di tempo regolari, ma lo studio viene messo in dubbio da alcuni test empirici;
			\item \textit{tone languages} (cinese mandarino) e \textit{stress languages} (inglese), nella prima categoria il pitch determina anche il significato della parola: poi però c'è una via di mezzo che sono le \textit{pitch-accented langauges};
			\item \textit{stress lessica} e \textit{stress frasale};
			\item verranno trattate lingue stress accented (inglese, spagnolo, olandese, italiano...);
			\item vengono individuati essenzialmente due attori principali nella definizione di prominenza:
			\begin{itemize}
				\item pitch accent
				\item stress (frasale)
			\end{itemize}
		\end{itemize}
	
	\subsection{Analisi fonetico-acustica}
		\begin{itemize}
			\item tre misure:
			\begin{itemize}
				\item segmentazione ennunciato e misure di durata, lo studio si baserà su unità di tipo sillabico; è necessario identificare nuclei sillabici dato che hanno le stesse caratteristiche delle sillabe intere, ma queste ultime sono difficili da identificare nel parlato; algoritmo:
					\begin{enumerate}
						\item isolamento dei nuclei
						\item identificazione dei confini dei nuclei
						\item durata dei nuclei sillabici
					\end{enumerate}
				\item misure relative ai profili intonativi (\textbf{\textit{pitch}}); i suoni \textbf{sonori} sono prodotti esclusivamente dalle corde vocali e poi ci sono quelli \textbf{sordi}. La frequenza fondamentale risulta nulla durante qualli sordi quindi il profilo del pitch risulterà definito esclusivamente in corrispondenza dei suoni sonori;
				\item misure relative all'intensità o energia;
			\end{itemize}
		\end{itemize}
	
	\subsection{Identificazione dei fenomeni prosodici}
		\begin{itemize}
			\item ricordiamo: per riconoscere fenomeno prosodico c'è bisogno dello stress della sillaba (fenomeno a carico del nucleo sillabico), o la presenza di un pitch accent all'interno della sillaba analizzata;
			
		\begin{figure*}
			\begin{table}[H]
				\label{my-label}
				\begin{tabular}{l|c|c|c|c}
					Fenomeni percettivi & \multicolumn{4}{c}{Prominenza}                                                                                                                                                               \\ \hline
					Fenomeni prosodici  & \multicolumn{2}{c}{Stress}                                         & \multicolumn{2}{c}{Pitch accent}                                                                                       \\ 
					Fenomeni acustici   & durata & \begin{tabular}[c]{@{}c@{}}enfasi\\ spettrale\end{tabular} & \begin{tabular}[c]{@{}c@{}}movimenti\\ in F0\end{tabular} & \begin{tabular}[c]{@{}c@{}}intensità\\ globale\end{tabular}
				\end{tabular}
			\end{table}
		\end{figure*}
		\end{itemize}
	
\section{Lettura \cite{bib:prominence-by-acoustic-analyses}}
	\begin{itemize}
		\item come è possibile definire la prominenza?
		\item qual è il miglior dominio di promineneza in acustica? sillabi;
		\item la prominenza è un fenomeno discreto o continuo? fenomeno continuo o multilivello;
		\item quali sono i parametri che combinati spiegano la prominenza? 
		\item ci sono parametri universali inter-linguistici?
		\item meglio gli approcci rulebase o machine learning?
		\item come possono essere valutati correttamente i sistemi di valutazione automatica?
	\end{itemize}
	\theoremstyle{plain}
	\newtheorem{definition}{Definizione}
	\begin{definition}[Prominenza]\label{def:prosodic-prominency}
		La prominenza prosodica è un fenomeno percettivo, continuo nella sua natura, che enfatizza alcune unità linguistiche e segmentali rispettivamente al contesto che li circonda, e viene supportata da una complessa interazione fra parametri prosodici e fonetico-acustici.
	\end{definition}

	Si può dare una prima parziale formalizzazione della prominenza:
	\begin{equation}
		\label{eq:prom}
		Prom^i = FA^i + PA^i
	\end{equation}
	$FA$ indica l'accento di forza, mentre $PA$ il pitch accent, il tutto all'$i$-esimo segmento.

\section{Lettura \cite{bib:prominence-detection-italian}}
	Vengono usate le (a partire da CRF che captano solo funzioni lineari) CNF, LDCRF, e le LDCNF. Questi tipi di PGM (Probabilistic Graphical Model) sono in grado di maneggiare sequenze di input output facendo previsioni su quella di output, considerando le configurazioni d input.
	\begin{itemize}
		\item l'$i$-esimo nodo in input prende un vettore di features;
		\item l'$i$-esimo nodo in output assegna un label, condizionato dai relativi vettori di features in input.
	\end{itemize}

	Vengono spiegati i vari tipi di \texttt{PGM} e il modo in cui è stato creato il corpus. Risultati ottenuti.

\section{Concetti Base}
	\subsection{Reti Deep: LSTM}
		Come descritto in \cite{bib:prominence-by-acoustic-analyses}, i principali vantaggi di usare metodi machine learning sono:
		\begin{itemize}
			\item approccio induttivo, il modello è appreso dai dati;
			\item classificazione rapida;
			\item rilevatori/classificatori altamente precisi.
		\end{itemize}
		Gli svantaggi invece consistono in:
		\begin{itemize}
			\item necessità di grossi corpora annotati per apprendere;
			\item i sistemi rischiano di essere legati allo specifico corpora o alla lingua usata durante la fase di apprendimento;
			\item i modelli prodotti sono tipicamente \textit{a scatola chiusa}, non c'è modo di estrarre informazioni linguistiche utili.
		\end{itemize}
		
\begin{thebibliography}{99}	
	\bibitem{bib:fenomeni-prosodici-prominenza}
		Fabio Tamburini,
		\newblock \emph{Fenomeni Prosodici e Prominenza: Un Approccio Acustico},
		2005.
	
	\bibitem{bib:prominence-detection-italian}
		Fabio Tamburini, Chiara Bertini, Pier Marco Bertinetto,
		\newblock \emph{Prosodic prominence detection in Italian continuous speech using probabilistic graphical models},
		2014
		
	\bibitem{bib:prominence-by-acoustic-analyses}
		Fabio Tamburini,
		\newblock \emph{Automatic Detection of Prosodic Prominence by Means of Acoustic Analyses},
		2015
\end{thebibliography}

\end{document}
